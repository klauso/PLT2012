\documentclass[a4paper,12pt]{article}

% Use the Classic Thesis style
%
% \usepackage[nochapters]{classicthesis}

\usepackage[T1]{fontenc}
\usepackage[utf8]{inputenc}

% Use text fonts
%
\usepackage{lmodern}  % the Latin Modern fonts which is an enhanced version of the Computer Modern fonts
% \usepackage{tgschola}  % the Gyre Schola fonts which is an enhanced version of the New Centrury Schoolbook fonts

% Use math fonts
%
% \usepackage{concmath}  % the Concrete Math fonts
% \usepackage{euler}  % the Euler Math fonts
% \usepackage{kerkis}  % the Kerkis fonts

% Some font packages that provide math font may override the text font, the
% following three commands reset it to the default.
%
% For the Latin Modern fonts
%
% \renewcommand{\rmdefault}{lmr}
% \renewcommand{\sfdefault}{lmss}
% \renewcommand{\ttdefault}{lmtt}
%
% For the Gyre Schola fonts
%
% \renewcommand{\rmdefault}{qcs}

\usepackage{amsmath}

% Use the MH bundle
%
% \usepackage{mhsetup}
% \usepackage{mathtools}
% \usepackage{mathstyle}
% \usepackage{breqn}
% \usepackage{empheq}
% \usepackage{flexisym}

% Use AMS-style theorem
%
\usepackage{amsthm}
% \theoremstyle{definition}
% \newtheorem{theorem}{Theorem}

% Use MATH formulas in PARagraph mode Typesetting Inference Rules
% 
\usepackage{mathpartir}

% \usepackage{tikz}
% \usepackage{tikz-qtree}
% \usepackage{bigfoot}

% New commands
%
\newcommand{\term}[1]{\textsf{#1}}
\newcommand{\NatT}{\textbf{N}}
\newcommand{\rarr}{\rightarrow}
\newcommand{\lamb}{\lambda}
\newcommand{\zero}{\textbf{0}}
\renewcommand{\succ}{\textbf{succ}}
\newcommand{\pred}{\textbf{pred}}
\newcommand{\isze}{\textbf{iszero}}
\newcommand{\BlnT}{\textbf{B}}
\newcommand{\true}{\textbf{true}}
\newcommand{\fals}{\textbf{false}}
\newcommand{\ifkw}{\textbf{if}}
\newcommand{\thkw}{\textbf{then}}
\newcommand{\elkw}{\textbf{else}}
\newcommand{\UniT}{\textbf{U}}
\newcommand{\unit}{\textbf{unit}}
\newcommand{\askw}{\textbf{as}}
\newcommand{\letk}{\textbf{let}}
\newcommand{\inkw}{\textbf{in}}
\newcommand{\fstp}{\textbf{fst}}
\newcommand{\sndp}{\textbf{snd}}
\newcommand{\inlh}{\textbf{inl}}
\newcommand{\inrh}{\textbf{inr}}
\newcommand{\case}{\textbf{case}}
\newcommand{\ofkw}{\textbf{of}}
\newcommand{\Rarr}{\Rightarrow}
\newcommand{\fixp}{\textbf{fix}}

\usepackage{stmaryrd}  % the St Mary’s Road symbol font

\newcommand{\atangles}[1]{\langle#1\rangle}
\newcommand{\atparens}[1]{(#1)}
\newcommand{\atbracks}[1]{[#1]}
\newcommand{\atbraces}[1]{\{#1\}}
\newcommand{\atfloors}[1]{\lfloor#1\rfloor}
\newcommand{\atceils}[1]{\lceil#1\rceil}
\newcommand{\atucorners}[1]{\ulcorner#1\urcorner}
\newcommand{\atlcorners}[1]{\llcorner#1\lrcorner}
\newcommand{\atlparens}[1]{\llparenthesis#1\rrparenthesis}
\newcommand{\atlbracks}[1]{\llbracket#1\rrbracket}
\newcommand{\atbars}[1]{|#1|}
\newcommand{\atbbars}[1]{\|#1\|}

\newcommand{\inangles}[1]{\langle#1\rangle}
\newcommand{\inparens}[1]{\left(#1\right)}
\newcommand{\inbracks}[1]{\left[#1\right]}
\newcommand{\inbraces}[1]{\left\{#1\right\}}
\newcommand{\infloors}[1]{\left\lfloor#1\right\rfloor}
\newcommand{\inceils}[1]{\left\lceil#1\right\rceil}
\newcommand{\inucorners}[1]{\left\ulcorner#1\right\urcorner}
\newcommand{\inlcorners}[1]{\left\llcorner#1\right\lrcorner}
\newcommand{\inlparens}[1]{\left\llparenthesis#1\right\rrparenthesis}
\newcommand{\inlbracks}[1]{\left\llbracket#1\right\rrbracket}
\newcommand{\inbars}[1]{\left|#1\left|}
\newcommand{\inbbars}[1]{\left\|#1\right\|}

\newcommand{\term}[1]{\textsf{#1}}
\newcommand{\warn}[1]{\textcolor{red}{#1}}

\newenvironment{grammar}
 {\begin{center}\small\begin{tabular}{rrclr}}
 {\end{tabular}\normalsize\end{center}}
\newcommand{\prodhead}[5]{#1 & #2 & #3 & #4 & #5 \\}
\newcommand{\prodrule}[3]{      & & #1 & #2 & #3 \\}



% Front elements
%
\title{
 Programming Languages and Types \\~\\
 \textbf{Homework 12}
}
\author{
 Yi Dai
}

\begin{document}

\maketitle

\section{Simply-Typed $\lambda$-Calculus}

\subsection{Typing Derivation}

Tell whether each of the following terms in the simply-typed $\lamb$-calculus with all the extensions 
introduced in the lecture is well-typed.  If it is, give a typing derivation for it; if not, give the
reason.  For very large terms, you can name their sub-terms and type them individually.

\begin{enumerate}
 \item $\pred\ (\succ\ \fals)$
 \item $\lamb f : \NatT \rarr \NatT . \lamb n : \NatT . f\ (f\ (\succ\ n))$
 \item $\ifkw\ (\isze\ (\succ\ \zero))\ \thkw\ \true\ \elkw\ \zero$
 \item $\{tru = \succ\ \zero, tru = \true\}\ \askw\ \{tru : \BlnT, one : \NatT\}$
 \item $\letk\ b = \fals\ \inkw\ (\isze\ b)$
 \item $\letk\ p = (\zero, \succ\ \zero)\ \inkw\ (\sndp\ p, \fstp\ p)$
 \item $\case\ (\inlh\ \zero)\ \ofkw\ \inlh\ x \Rarr \fals\ |\ \inrh\ x \Rarr \true$
 \item
  \begin{align*}
   \fixp\ (& \lamb\ fise : (\NatT \rarr \BlnT) \rarr (\NatT \rarr \BlnT)\ . \\
           & \quad \lamb\ n : \NatT\ . \\
           & \quad \quad \ifkw\ (\isze\ n) \\
           & \quad \quad \quad \thkw\ \true \\
           & \quad \quad \quad \elkw\ \ifkw\ (\isze\ (\pred\ n)) \\
           & \quad \quad \quad \quad \quad \thkw\ \fals \\
           & \quad \quad \quad \quad \quad \elkw\ fise\ (\pred\ (\pred\ n))\ )
  \end{align*}
\end{enumerate}

\subsection{Programming with Extensions}

\begin{enumerate}
 \item Complete the addition function $add$ in the simply-typed $\lamb$-calculus extended with Peano numbers
  (\zero\ and \succ) and fixed point operator \fixp.\footnote{During the exercise session, I gave the wrong
  type $(\NatT \rarr \NatT \rarr \NatT)$ to the variable that is to be bound to the fixed point.  Please
  refer to the exercise sheet \texttt{ex12.pdf}.  In this homework exercise, I have given the type for
  $fadd$, to remind of the mistake I made.}
  \begin{align*}
   \fixp\ (& \lamb\ fadd : (\NatT \rarr \NatT \rarr \NatT) \rarr (\NatT \rarr \NatT \rarr \NatT)\ .\ ?)
  \end{align*}
\end{enumerate}

\section{System-$\mathcal{F}$}

\subsection{Parametric Polymorphism}

\subsection{Typing Church-Encodings}

\end{document}
