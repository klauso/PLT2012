\documentclass[a4paper,12pt]{article}

% Use the Classic Thesis style
%
% \usepackage[nochapters]{classicthesis}

\usepackage[T1]{fontenc}
\usepackage[utf8]{inputenc}

% Use text fonts
%
\usepackage{lmodern}  % the Latin Modern fonts which is an enhanced version of the Computer Modern fonts
% \usepackage{tgschola}  % the Gyre Schola fonts which is an enhanced version of the New Centrury Schoolbook fonts

% Use math fonts
%
% \usepackage{concmath}  % the Concrete Math fonts
% \usepackage{euler}  % the Euler Math fonts
% \usepackage{kerkis}  % the Kerkis fonts

% Some font packages that provide math font may override the text font, the
% following three commands reset it to the default.
%
% For the Latin Modern fonts
%
% \renewcommand{\rmdefault}{lmr}
% \renewcommand{\sfdefault}{lmss}
% \renewcommand{\ttdefault}{lmtt}
%
% For the Gyre Schola fonts
%
% \renewcommand{\rmdefault}{qcs}

\usepackage{amsmath}

% Use the MH bundle
%
% \usepackage{mhsetup}
% \usepackage{mathtools}
% \usepackage{mathstyle}
% \usepackage{breqn}
% \usepackage{empheq}
% \usepackage{flexisym}

% Use AMS-style theorem
%
\usepackage{amsthm}
% \theoremstyle{definition}
% \newtheorem{theorem}{Theorem}

% Use MATH formulas in PARagraph mode Typesetting Inference Rules
% 
\usepackage{mathpartir}

% New commands
%
\newcommand{\term}[1]{\textsf{#1}}

\usepackage{stmaryrd}  % the St Mary’s Road symbol font

\newcommand{\atangles}[1]{\langle#1\rangle}
\newcommand{\atparens}[1]{(#1)}
\newcommand{\atbracks}[1]{[#1]}
\newcommand{\atbraces}[1]{\{#1\}}
\newcommand{\atfloors}[1]{\lfloor#1\rfloor}
\newcommand{\atceils}[1]{\lceil#1\rceil}
\newcommand{\atucorners}[1]{\ulcorner#1\urcorner}
\newcommand{\atlcorners}[1]{\llcorner#1\lrcorner}
\newcommand{\atlparens}[1]{\llparenthesis#1\rrparenthesis}
\newcommand{\atlbracks}[1]{\llbracket#1\rrbracket}
\newcommand{\atbars}[1]{|#1|}
\newcommand{\atbbars}[1]{\|#1\|}

\newcommand{\inangles}[1]{\langle#1\rangle}
\newcommand{\inparens}[1]{\left(#1\right)}
\newcommand{\inbracks}[1]{\left[#1\right]}
\newcommand{\inbraces}[1]{\left\{#1\right\}}
\newcommand{\infloors}[1]{\left\lfloor#1\right\rfloor}
\newcommand{\inceils}[1]{\left\lceil#1\right\rceil}
\newcommand{\inucorners}[1]{\left\ulcorner#1\right\urcorner}
\newcommand{\inlcorners}[1]{\left\llcorner#1\right\lrcorner}
\newcommand{\inlparens}[1]{\left\llparenthesis#1\right\rrparenthesis}
\newcommand{\inlbracks}[1]{\left\llbracket#1\right\rrbracket}
\newcommand{\inbars}[1]{\left|#1\left|}
\newcommand{\inbbars}[1]{\left\|#1\right\|}

\newcommand{\term}[1]{\textsf{#1}}
\newcommand{\warn}[1]{\textcolor{red}{#1}}

\newenvironment{grammar}
 {\begin{center}\small\begin{tabular}{rrclr}}
 {\end{tabular}\normalsize\end{center}}
\newcommand{\prodhead}[5]{#1 & #2 & #3 & #4 & #5 \\}
\newcommand{\prodrule}[3]{      & & #1 & #2 & #3 \\}



% Front elements
%
\title{
 Programming Languages and Types \\~\\
 \textbf{Homework 10}
}
\author{
 Yi Dai
}

\begin{document}

\maketitle

\section{Abstract Syntax vs. Concrete Syntax}

\begin{enumerate}
 \item Give the abstract syntax for the language ABE (Arithmetic-Boolean Expressions)
  presented in the lecture, in BNF and in Scala.

 \item Give a concrete syntax other than infix notation for ABE. That is, choose either
  prefix notation or postfix notation.
\end{enumerate}

\section{Inductive Definitions and Rule Induction}

\begin{enumerate}
 \item Give an inductive definition for the ternary relation $Rem$:
  \[
    (m, n)\ Rem\ r\ \text{if}\ r\ \text{is the remainder when}\ n\ \text{is divided by}\
    m, \text{where}\ m \ne 0.
  \]
  You can assume the availability of other arithmetic operations or relations.

 \item Use the inductive definition for Rem, prove that
  \[
    (m, n)\ Rem\ r_1\ \text{and}\ (m, n)\ Rem\ r_2\ \text{implies}\ r1 = r2.
  \]
\end{enumerate}

\section{Evaluation Semantics vs. Reduction Semantics}

\begin{enumerate}
 \item Give the evaluation semantics for ABE.

 \item Implement both the evaluation semantics and the multi-step reduction semantics for
  ABE. Show that the two implementations give the same results for the same expressions by
  a reasonable number of tests.

 \item{} [\emph{optional}] Prove that the evaluation semantics coincides with the multi-step
  reduction semantics, that is,
  \[ \forall e \in Exp, e \Longrightarrow v\ \text{if and only if}\ e \longrightarrow^{*}
      v.
  \]
\end{enumerate}

\end{document}

